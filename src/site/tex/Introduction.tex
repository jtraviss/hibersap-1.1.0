%!TEX root = HibersapReference.tex
\chapter{Introduction to Hibersap}

  Hibersap helps developers of Java applications to call business logic in SAP backends. 
  It defines a set of Java annotations to map SAP function modules to Java classes as well as a small, clean API
  to execute these function modules and handle transaction and security aspects. 
  Hibersap's programming model is quite similar to those of modern O/R mappers, significantly speeding up the 
  development of SAP interfaces and making it much more fun to write the integration code.
  \\ \\
  Under the hood, Hibersap either uses the SAP Java Connector (JCo) or a JCA compatible resource adapter 
  to communicate with the SAP backend. While retaining the benefits of JCo and JCA like transactions, 
  security, connection pooling, etc., developers can focus on writing business logic because the need for boilerplate
  code is largely reduced.
  \\ \\
  Hibersap can either be configured programmatically or by providing an XML file. Switching between JCo and JCA
  is a sole matter of configuration, the program code remains the same. This makes it possible to execute
  integration tests via JCo while using a resource adapter in the production environment.
  \\ \\
  Regarding data type conversion from ABAP to Java types, Hibersap per default uses the conversion as is done by 
  JCo resp. JCA. Custom converters may be used to implement special conversion logic. Hibersap will then call the
  conversion code on-the-fly, before and after calling the function module in SAP. 
  \\ \\
  Hibersap may be configured to use Bean Validation (JSR 303) to validate field values according to the standard
  Bean Validation annotations.
  \\ \\
  If the function module defines a standard Return structure or table, Hibersap is able to automatically detect 
  an error state and throw a SapException which includes the information returned by SAP.
  \\ \\
  For Java EE applications it is recommended to use a resource adapter since it integrated seamlessly with  
  Java EE containers. Using the Hibersap EJB tools makes it very easy to call SAP from EJB methods including
  such nice features like Container Managed Transactions and Container Managed Security. In a managed environment
  calls to SAP functions may even be part of distrubited transactions.  
  \\ \\
  For advanced use cases, there are two types of interceptors that work on different levels of the call stack.
  \\ \\ \\
  In chapter \ref{cha:QuickStart} you'll find out how easy it is to use Hibersap.
